\section{National Importance}
The UK Government's Cyber Security Strategy identifies the need to
strengthen resistance to cyber attacks and ensure service
availability. Targeted attacks can impact availability of information
networks --- an increasingly essential component of national critical
infrastructure and businesses who are coming under DDoS attacks.
According to Arbor Networks\footnote{Arbor DDoS Threat
  Report,~\url{http://www.arbornetworks.com/corporate/blog/4922-q2-key-findings-from-atlas}},
1 in 5 companies in the UK have already fallen victim to a DDoS
attack. Further, according to the Intelligence and Security Committee
annual report 2012--2013, many government departments have come under
targeted DDoS attacks.

\cut{Targeted attacks on networks is a specific form of Advanced
Persistent Threats (APTs), with the potential to cause widespread
damage, since most computer applications are based on the assumption
of highly available information networks --- an increasingly essential
component of military operations, government communication systems,
and businesses.   The ability of these networks to cope with
  increasing demands on scale and performance, to perform efficiently
  in harsh operating environments, and to adapt to new applications
  and threats lies in the highly sophisticated and complex software
  making up their design. Unfortunately, the complexity of this
  software makes it prone to vulnerabilities and misconfigurations
  that can be targeted by attackers to gain access to network data, or
  disable network operation.  This threat is growing increasingly
  severe with modern malware coordinating operations across a
  distributed network.}

Our programme offers the potential for UK research to be at the
forefront of future network defences. A possible outcome is that SDN
routers could be shown to be more secure than expensive
special-purpose hardware routers, maybe decreasing the costs of
network infrastructure by an order of magnitude.  Juniper Networks
supplies a significant fraction of routing equipment both nationally and
internationally, so we are in a position to positively influence
the network security environment in the UK and beyond.

\cut{
 Our user-partner \fcomment{there are more than one, we could just name them here?} is a router
manufacturer whose products form a significant part of the UK's information
infrastructure.}



\section{Academic Impact}

SDN is an emerging trend in the design of % secure da: WE GIVE NO EXAMPLES??
flexible networks. Our research will benefit those who build on or
take inspiration from our advances. For example, our foundational work
on a framework for analysing topology transformation strategies may
influence other scientists designing resilient
networks. Our work on covert timing channels will be useful to those designing
better tools for debugging the control plane. Our investigations of
formal security verification may provide cross-influences with
researchers working on rigorous models of configuration correctness. %% was: security

To ensure these influences happen, we will
provide web content, publications, and attend key meetings. We
aim to publish in top conferences such as SIGCOMM, Usenix Security, and
NDSS, as well as relevant journals.  During the project, we plan
to convene two specialised workshops on Security and SDN.
% SDN security and verification.   DA: Eh?? Two workshops on 1/4 of project?

We expect rich interaction between the Universities and the industrial partners
in the consortium.  We plan a specific external collaboration with {\bf
Prof. Jennifer Rexford (Professor of Networking) from Princeton University}, who
is at the forefront of academic research efforts on SDN. She will help set the
research direction for the dynamic control plane. We will use her work on SDN
primitives to build topology transformations needed for security.  We also plan
to combine her group's work with our previous work on resilience, and develop
migration techniques that function in the presence of malicious controllers.

% (2) {\bf Dr. Andrew W. Moore from University of
%   Cambridge}, who jointly leads the NetFPGA programme. We will collaborate
% on measurement characterisation and use transferable results to evolve
% switch primitives.
%\vspace{-1pt}
%
%\vspace{-2pt}
%{\bf Andrew Moore, Cambridge University} is an expert in SDN based routers for specialised applications.\\
%\vspace{-2pt}
%{\bf Ross Anderson, Cambridge University} is an expert on Security Engineering with interest in developing dependable networks based on SDN.\\
%% \vspace{-2pt}
% {\bf Add other researchers here.}



%\scomment{3 pages max from here till the end}
