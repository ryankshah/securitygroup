\documentclass[11pt]{article}
\usepackage{common}

\begin{document}
    \maketitle
    \setcounter{section}{6}
    \section{External research funding and impact of projects}

    \makeatletter

    \define@key{grant}{title}{\def\mytitle{#1}}
    \define@key{grant}{principalinvestigator}{\def\myprincipalinvestigator{#1}}
    \define@key{grant}{otherinvestigators}{\def\myotherinvestigators{#1}}
    \define@key{grant}{dates}{\def\mydates{#1}}
    \define@key{grant}{fundingagency}{\def\myfundingagency{#1}}
    \define@key{grant}{value}{\def\myvalue{#1}}
    \define@key{grant}{mainoutcomes}{\def\mymainoutcomes{#1}}
    \define@key{grant}{summary}{\def\mysummary{#1}}

    \presetkeys{grant}{}{title={No title},principalinvestigator={None},otherinvestigators={}}

    
    \newcommand\grant[1][]{%
      \setkeys{grant}{#1}
        \noindent
        {\bf \mytitle}\\
        {\bf Principal investigator:} \myprincipalinvestigator\\
        \ifthenelse{\not\equal{\myotherinvestigators}{}}{{\bf Other investigators:} \myotherinvestigators\newline}{}
        {\bf Dates:} \mydates\\
        {\bf Funding agency:} \myfundingagency\\
        {\bf Financial value:} \myvalue\\%
        \ifthenelse{\not\equal{\mymainoutcomes}{}}{{\bf Main outcomes:} \mymainoutcomes\\}{}
        \ifthenelse{\not\equal{\mysummary}{}}{{\bf Summary:} \mysummary\\}{}
        %\ifthenelse{\not\equal{\mynotes}{}}{{\it\mynotes\newline}}{}
    }
    \newcommand\epsrc{ (EPSRC contribution)}
    \newcommand\UoS[1]{ (\textsterling #1 for the University of Strathclyde)}
    \newcommand\ofwhichUoS[1]{ of which \textsterling #1 for the University of Strathclyde}


    \grant[title = {Robustness-as-evolvability (Competitive)},
        principalinvestigator= {Shishir Nagaraja},
        dates                = {04/2015 -- 03/2019},
        fundingagency        = {EPSRC},
        value            ={\pounds 790,000 \epsrc \UoS{294000} Lead institution},
        summary={%
          Dr. Nagaraja leads an international consortium of five universities  and five industry partners on SDN security. This project explores novel architectures for secure control in IoT, SDN, and NFV environments. These environments present unique management challenges due to the sheer scale, complexity, dynamicity, and the substantial attack surface that must be defended. This project has developed a novel dynamic quorum primitive and applied it to build a scalable distributed control architecture that can defend against insider attackers. The project involves the extensive application of machine learning primitives at all levels of router design –- from smart caching algorithms in switches to adversarial measurement techniques to adversary-resistant controllers. },
        mainoutcomes={Dr. Nagaraja and his team published three peer-reviewed papers with two more papers in submission. With the support of UK Govt. and industry, we created the  ENACT and RANSOM testbeds as part of this project. The testbed is used by the team and the industrial partners on the project. In terms of impact, Brocade is currently considering the quorum control architecture for productisation, while Juniper's product management is considering the adoption of SDN cache defences within their EX series switches.}]

        \grant[%
      title ={CREATe (Competitive)},
      principalinvestigator = {Lilian Edwards},
      dates = {10/2012--08/2017},
      fundingagency = {AHRC,ESRC,EPSRC},
      value = {\textsterling 4,169,477\epsrc \UoS{206,025}},      
      summary = { Prof. Edwards as Deputy Director, co-leads the CREATe
        consortium of 7 universities and 80-plus industry, public
        sector and policy partners, for a programme of more than 80
        work packages over 4 years, funded for £5m, plus £3m industry
        contribution. Prof. Edwards has general responsibility for the
        digital security side of the centre and also in charge of 7
        funded packages, running for various periods over the four
        years, including work on digital copyright enforcement and its
        security economics; adversarial data mining; privacy of
        digital assets; data-protection in networked technologies;
        privacy and security in smart cities; privacy, trust, and
        confidence in digital services; Privacy and Disclosure on
        Twitter feeds.},
      mainoutcomes={Prof. Edwards and her team published six peer-reviewed papers on legal aspects of cybersecurity}]

        \grant[title ={ SHAWN: Secure High availability Avionics Wireless
          Networks (Competitive) }, principalinvestigator = {R Atkinson},
        otherinvestigators = {I Andonovic, C Michie, D
          Harle}, dates = {1/14--06/17}, fundingagency = {Technology
          Strategy Board}, value = {\textsterling 723,965 UKRI
          contribution \UoS{162,724}}, summary = {This project focused
          on the security of wireless networks used to interconnect
          different components on aircrafts. While wireless brought
          weight savings and greater flexibility in these systems,
          contributing to reduced fuel burn and less instances of
          unscheduled maintenance, the use of insecure wireless
          technology can compromise flight safety. Instead of
          confidentiality, the core requirement was availability. This
          project had two objectives which were both achieved:
          developing aircraft wireless networks that were (a)
          intrusion-tolerant and (b) jamming resistant. The core
          innovation from Strathclyde was the development of security
          mechanisms that leverage channel redundancy to achieve
          denial-of-service resistance.},
        mainoutcomes={The project
          led to the development of advanced security features within
          GE's aircraft network systems including the design of secure
          antennas to resist DoS attacks, eavesdropping attacks, and malicious
          corruption of data on the network.}]

          \grant[% Three security projects: TCS-002 Cyber Best Practice, TCS-005 Security across organisational boundaries, TCS-006 Cyber security incident response, June 2017, 18 months PI
      title ={Power Systems Security (Competitive)},
      principalinvestigator = {J Irvine},
      otherinvestigators = {R Atkinson},
      dates = {06/2017--01/2019},
      fundingagency = {Innovate UK},
      value = {\textsterling 200,000},      
      summary = {This project will survey the UK utility sector to
        identify from a landscape of potential incidents to identify
        the capacity needed to respond and recover to cyber security
        events. This will include considering the necessary functions
        that need to be maintained during an attack and assessing the
        levels of incident handling that can be carried out by
        DNOs. An incident response framework will be developed that
        enables identification of the capabilities and redundancy
        back-up that are required to withstand an attack and manage
        recovery. This project considers mechanisms to encourage the
        achievement of cyber security across organisational boundaries
        and throughout supply chains in electricity distribution. Due
        to the growing attack surface and increase in volume of
        attacks, a more efficient deployment and ongoing assurance of
        security controls will be investigated. The project also
        considers the security of vendor equipment and vendor
        interactions with operational systems through the design of
        secure and verifiable software environments which can be
        audited.  },
      mainoutcomes={The main outcomes with be the
        implementation of a secure PC environment that can be used in
        an OT environment. Through PNDC (see
        2-description-of-applicant.pdf) member links to ENA, this will
        be disseminated in the UK utility sector more generally, and
        through Strathclyde's membership of EE-ISAC, utilities in
        Europe.}]


          \grant[%
      title ={Privacy-preserving computation on the Cloud (Competitive)},
      principalinvestigator = {Sotirios Terzis},
      otherinvestigators = {Cheng Dong},
      dates = {04/13--03/17},
      fundingagency = {EPSRC DTG},
      value = {\textsterling 11,948},
      summary = {This project focused on scalable private
        set-intersection. A problem of longstanding interest within
        the security community; given a number of encrypted sets, how
        can a cloud operator (or a set of distributed participants)
        reliably compute the intersection of these sets without
        learning any information about non-intersecting elements of
        the sets.},
      mainoutcomes = {This project developed three
        security protocols for private set-intersection. The first
        protocol ensures confidentiality under a honest-but-curious
        adversary using homomorphic cryptosystems. The second protocol
        is a significant improvement in terms of security, as it
        assumes an active adversary, whilst also doing away with
        homomorphic cryptosystems resulting in a more efficient
        solution. The third protocol, adds verifiability and formal
        proofs of security. The project outputs were published in
        respectable peer-reviewed journals and conferences.}]
          
        
        
\end{document}
