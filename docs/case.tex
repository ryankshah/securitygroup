\documentclass[11pt]{article}
%\usepackage{common}
\usepackage[margin=2cm]{geometry}

% Get student number from section 5
%\importvars{5-students}
\def\numphdstudents{27}

%\title{Security Group}
\title{Case for Support:\\Academic Centre-of-Excellence in Cyber Security Research}

\author{University of Strathclyde}
\date{}

\begin{document}
\maketitle

\section{Expertise}
    The \emph{Security Group} at the University of Strathclyde consists of the
    following permanent academic members:
    \begin{description}
        \item[Dr.\ Shishir Nagaraja] Reader in Computer Security, who leads the group
        \item[Prof.\ Lilian Edwards] Professor of Internet Law and Policy
        \item[Dr.\ James Irvine] Reader
        \item[Dr.\ Robert Atkinson] Senior Lecturer
        \item[Dr.\ George Weir] Lecturer
        \item[Dr.\ Sotirios Terzis] Lecturer
%        \item[Dr.\ Robert Atkey] Lecturer
%        \item[Dr.\ Narisong Huhe] Lecturer
    \end{description}
    It also includes three staff researchers (Dr.\ Craig Hay, Dr.\ Greig Paul, Dr.\ Kinan Ghanem), two staff vacancies, and \numphdstudents \ PhD students.

    \subsection{Research expertise}
    The group leads research in the following areas:
    \begin{itemize}
    \item Network Security (Nagaraja, Terzis, Irvine, Atkinson)
    \item Privacy and Security (Nagaraja, Edwards, Terzis)
    \item Intrusion detection(Terzis, Nagaraja, Atkinson)
    \item Human factors and Socio-technical security (Weir and Terzis)
    \item Security governance (Edwards)
    \item Machine Learning and Security (Nagaraja, Atkinson)
 %   \item Analysis and verification of systems (Atkey)
    \item Network resilience (Irvine)
 %   \item Complex network analysis (Huhe)
    \item Botnets and Malware (Nagaraja)
    \end{itemize}


    \section{Background}

    \subsection{Development in the last five years}
    The security group has been active since 2006. The group has
    steadily grown to its current size of eight permanent members,
    five postdocs and \numphdstudents \ PhD students over a period of
    ten years. The group has broad aims which  covers
    significant aspects of the field of computer security.  Its
    strength is in carrying out practical impact-led security research
    in close collaboration with real-world users.
        
    \subsection{Invited talks}
    Members of our group are frequently invited to give keynote talks.
    (nine in 2017, six in 2016). 
    
    \subsection{Facilities}
    Due to its practical focus, the group has built and run a number
    of security testbeds.

    The {\bf Experimental Network ArChitectures Testbed (ENACT)}
    is a specialist testbed for enacting a variety of
    network-infrastructure scenarios, enabling networked systems
    research from low-level physical wiring to network protocols and
    applications, via a software-defined networking (SDN)-capable
    network testbed. ENACT is currently composed of 1056 switch ports
    and 96 server ports, using Pica8 3290N switches and Juniper 4500EX
    switches.\\

    The {\bf Ransom Architectures for Network and Systems
      Opportunistic Malware (RANSOM) Testbed} enables malware and
    vulnerability research for opportunistic malware on a cluster farm
    of desktop, mobile, and server class hardware. RANSOM is composed
    of an exclusive datacentre that hosts up to 100k virtual machines
    on a 40Gbps Mellanox network. The farm runs malware honeypots,
    experimental malware defences, and offensive technologies.\\
    
    The {\bf Power Network Demonstration Centre (PNDC) Testbed} is an
    AC electricity grid emulator that enables research on experimental
    architectures for secure power systems. It helps enact a variety
    of scenarios from bidirectional 500kVa to 1MVA power systems, thus
    providing a facility for studying the security of power systems in
    the face of adversaries. For instance, to create adversarial loads
    in power systems and adversarial interference in phase balancing
    mechanisms in high-voltage networks. It is composed of expandable
    Triphase modular power system hardware within two substations
    connected by fibre-optic cables, scada interface equipment
    connected via Cisco connected routers and managed via
    software-defined controllers.
    
    \subsection{Recent investments}
    Since 2006 the security group has acquired over
    \pounds 3.88 million % keep updated
    in research funding, from EPSRC, VMWare, Samsung, IBM, Juniper, Brocade, Scottish Government, among many others. The group currently participates in projects having a total value of   \pounds 9.886 million. % keep updated
    

    \section{National Importance}
The UK Government's Cyber Security Strategy identifies the need to
strengthen resistance to cyber attacks and ensure service
availability.  Cyber attacks continue to create a Tier 1 risk as judged in the National Security Risk Assessment 2015.  Cyber attacks have the  potential to cause widespread
damage, since most computer applications are based on the assumption
of highly available information networks --- an increasingly essential
component of military operations, government communication systems,
and businesses.   The ability of these networks to cope with
  increasing demands on scale and performance, to perform efficiently
  in harsh operating environments, and to adapt to new applications
  and threats lies in the highly sophisticated and complex software
  making up their design. Unfortunately, the complexity of this
  software makes it prone to vulnerabilities and misconfigurations
  that can be targeted by attackers to gain access to network data, or
  disable network operation.  This threat is growing increasingly
  severe with modern malware coordinating operations across a
  distributed network.

  {\textbf Academic impact:} Our centre offers the potential for UK
  research to be at the forefront of academic research in selected
  sub-areas of cybersecurity.  We expect rich interaction between UoS,
  other ACEs, and the wider cybersecurity community nationally and
  internationally. Particularly, via the secondment programme and the
  research workshops as part of the centre's activities.

\section{Objectives and Methodology}

Our vision for our Academic Centre of Excellence in Cybersecurity
Research is to ensure cybersecurity delivers on its potential to be
one of the key underpinning technologies that will drive our economic,
social and scientific development in the decades ahead.

The Strathclyde Cyber Security Centre of Excellence will be hosted within the Strathclyde Centre for Doctoral Training in Cybersecurity. This is a multi-disciplinary centre spanning the Department of Computing and Information Sciences, the Department of Electrical Engineering, and the Department of Law. The centre involves one professor (Edwards), two Readers (Nagaraja, Irvine), one Senior Lecturer (Atkinson), and two Lecturers in Computer Security (Weir,Sotirios). The centre has a dedicated member of staff (Dr. William Wallace) to manage its flagship knowledge-exchange program and the extensive links it has with industry, NGO, and government organisations.

The centre is internationally renowned for its systems security focused research that bridges systems security (Nagaraja, Terzis, Irvine, Edwards, Atkinson), behavioural (Weir, Terzis) and legal aspects (Edwards, Weir) of cybersecurity research. These permanent members of staff are supported by three postdoctoral research associates (Hay, Paul, Ghanem). This research has been funded from a variety of sources including research councils (EPSRC, ESRC, AHRC), the European Commission, and direct investment from security organisations. Since 2006, the centre has received total research income of £3.88m from these sources. In addition, the centre has a thriving PhD program with 12 students completing their theses between April 2013 and March 2018 and another 27 current PhD students. Most of the staff and students are principally housed in the Department of Computing and Information Sciences.

Four key principles underlie the centre's research ethos that distinguish it from other centres: First, its focus on the ubiquitous and transformative nature of cybersecurity research spanning science, engineering, business and the social sciences; Second, its intersectorality drawing on leading academics across these Faculties, industry (large multinationals, SMEs, and start-ups), and third sector organisations (e.g. health providers, local government, police and armed forces); Third, the scale of its intended impact based upon around 75 companies who are already part of the Strathclyde Knowledge Exchange ecosystem; and Fourth, its focus on delivering end-to-end solutions including a strong focus on the legal, ethical and regulatory frameworks governing the use of cybersecurity. 

Each aspect is crucial:
\begin{itemize}
\item Understanding the ubiquity of cybersecurity
research is vital in developing the full spectrum of cybersecurity
techniques as opposed to focussing on a limited sample of those
techniques as often happens elsewhere; 

\item  Intersectorality is vital
in ensuring cybersecurity feeds into and is informed by diverse
problems so as to gauge which techniques and approaches work best in
which domains; 

\item Focussing on impact at scale is vital in ensuring
  our research can tackle significant real world problems; and

\item Taking an end-to-end approach ensures we can not only perform core
cybersecurity-interventions, but can also better choose which
interventions are technically and ethically appropriate, and then
translate the ensuing results into benefits for end-users. Crucially,
research in the ethical, legal and regulatory framework will not only
mean our results are fit-for-purpose, but also they help society fully
embrace the benefits of cybersecurity. This ubiquitous and end-to-end
approach significantly broadens the appeal of our vision. Thus, the
Academic Centre for Excellence in Cybersecurity Research will be
managed by leading academics, industry, business professionals, and
policy makers who have been chosen to ensure expertise across the
Centre.
\end{itemize}
    
    
    Examples of the group's recent and ongoing work of
    this kind include:
    \begin{itemize}
    \item SDN security
    \item Malware and botnet detection
    \item Privacy-preserving computation
    \item Power networks security
    \item Massively parallel security defences with GPUs
    \item Privacy and digital anonymity in online social networks and cloud computing
    \item Copyright protection (adversarial) in machine-learning covering both legal and technological approaches
    \item Extremist content classification
    \item Cultural aspects of security
    \item Regulation of online intermediaries (Google, eBay, Facebook, etc.)
    \item Regulation of Robots, Autonomous Vehicles, and Digital Assets
    \end{itemize}

    
    The group's strengths are in
    \emph{network security} (Nagaraja, Irvine, Atkinson);
    \emph{access control} (Terzis);
  %  \emph{verification} (Atkey);
    \emph{cloud security} (Terzis, Edwards);
    \emph{analysing attacks} (Nagaraja, Weir);
    \emph{security design principles} (all);
    \emph{human factors in security} (Weir, Terzis);
    \emph{privacy, anonymity and security} (Edwards, Nagaraja);
    \emph{malware and intrusion analysis} (Nagaraja, Atkinson);
  %  \emph{information flow} (Nagaraja, Atkey);
    \emph{security and privacy law} {Edwards};
    \emph{software-defined networks} (Nagaraja); 
    \emph{cyber-physical systems} (Irvine, Atkinson)

    
    To ensure that cybersecurity delivers on its potential to be one
    of the key underpinning areas that will drive scientific,
    societal, and economic development, we have adopted a
    cross-disciplinary approach towards security research. This allows
    us to take a holistic view of security challenges; we work on
    foundational components (security protocols and privacy-preserving
    computation), to novel techniques (malware, SDN, power-networks, traffic analysis) that are sympathetic to the local context
    (for instance how cultural aspects influence information
    security). Crucially, research in the ethical, legal and
    regulatory framework will not only mean our results are
    fit-for-purpose, but also they help society fully embrace the
    benefits of cybersecurity. This ubiquitous and end-to-end approach
    significantly broadens the appeal of our vision.

    
    \subsection{Research culture}
    The centre carries out research collaboratively with other academic partners
    including Universities of Edinburgh, Newcastle, Glasgow, Abertay,
    Manchester, UCL, Kings College, University of Illinois at
    Urbana-Champaign (UIUC), Stanford University, UC Berkeley, and
    Princeton University.  The group has hosted numerous speakers and
    visiting scholars from these universities as well as from Bristol,
    Cambridge, and Oxford. Recent industrial speakers have included
    the Scottish Police, NPL, Morgan Stanley, NHS Scotland, Previse,
    Samsung, VMWare, Juniper, among others. The group regularly
    conducts workshops, organises monthly reading groups, and a
    hacking club. Members of the group are regularly invited as
    visiting professors. Nagaraja is Adjunct Professor at UIUC and previously Visiting Professor at EPFL; Edwards is Visiting Professor at U. Edinburgh, and previously at UC Berkeley and Stanford University; and, Weir is Adjunct Professor at Simon-Fraser University.


%    Atkey works with banks     and trading firms to detect potential concurrency attacks in      distributed computing environments. 

    \subsection{Planned Activities 2018--2022}
    As part of making Strathclyde ``a place of useful security
    research'', the centre will initiate a major expansion of our outreach
    programme by way of creating a Strathclyde Cybersecurity
    Network. The main thematic area of focus will be secure industrial
    informatics. This will involve a number of workshops as well as
    site visits to organisations in sectors whose operational security
    is critical to the integrity of UK national infrastructure. We
    will initially leverage the Strathclyde knowledge-exchange
    ecosystem but we will go well beyond that. The goal is to take
    research into the core of industrial processes and bring
    practitioner-wisdom into our research. The security group will
    complete the process of of creating the network of organisations
    by 2019. This network will include companies (driven by Industry
    4.0 priorities), NGOs, governments, and policy think tanks who
    face complex cybersecurity challenges that demand a collaborative
    response.

    The second major activity will be a secondments programme.  The
    group's ethos is to carry out research of importance to society
    including industry and government. Because of this ethos, the
    group is naturally financially self-sustaining: it shifts its
    focus in line with national importance. In line with this, we need to develop close working relationships with a select group of organisations. We will ensure this by hosting them within the ACE-CSR or by providing opportunities for Strathclyde staff and students to spend a period of secondment at stakeholder sites. This will maximise the potential of transferring knowledge, tools, and techniques from the ACE-CSR to the industry and practice.

    Third, we plan to hold a number of UK all-hands meetings at
    periodic internvals which will enable direct meetings with
    stakeholders and organisations, these are designed to maximise
    impact from academia to industry and vice-versa.

    Fourth, we have grown by about four people per year for the last
    five years, and we anticipate further growth of three people per
    year for another four years, reaching a steady-state size of 45
    people.

    
    \subsection{Planned impact}
    
The centre's activities are focused on a number of organisations in industry, policing, and government who are interested in cybersecurity. The centre's objective is to maximise impact from ACE-CSR funds and status. As such a number of activities have been planned with this objective in mind. The creation of a network on the cybersecurity of industrial informatics will lead to the creation of a focused group of  stakeholders. Further, the secondments programme  builds upon the initial networking phase and is designed to develop close working relationships  via reciprocal staff exchanges. This will maximise the knowledge transfer from the ACE-CSR to practitioners as well as be a pathway for ensuring that practitioner concerns are adequately reflected in the research we pursue.

The ACE will not operate in isolation and there is an extensive network of industry, practice and community contacts open to us through existing projects. Further, the ACE will have access to 75+ companies within the Strathclyde Knowledge Exchange ecosystem who have said to us that they are keen to engage with us on cybersecurity.  All of these will provide invaluable pathways to impact for research. The ACE-CSR status will enable us to harmonise these various contacts into a systematic approach for maximising impact from our cybersecurity research. 


    \section{Management}
    The PI will provide leadership to the centre and take day-to-day
    charge of the centre's activities and harmonise the efforts by
    members into a systematic approach for maximising impact from our
    cybersecurity research.  The VC's office will provide oversight on
    the centre's activities and on its strategic direction.

    
\end{document}
